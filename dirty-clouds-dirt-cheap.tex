\documentclass[aspectratio=169,11pt,hyperref={colorlinks=true}]{beamer}
\usetheme{boxes}
\setbeamertemplate{navigation symbols}{}
\definecolor{openstack}{RGB}{149,0,4}
\setbeamercolor{titlelike}{fg=openstack}
\setbeamercolor{structure}{fg=openstack}
\hypersetup{colorlinks,urlcolor=openstack}
\setbeamertemplate{footline}[frame number]
% Inserting graphics
\usepackage{graphicx}
% Side-by-side figures, etc
\usepackage{subfigure}
% Code snippits
\usepackage{listings}
\usepackage{lmodern}
% Color stuff
\usepackage{color}
\usepackage{amsmath}
\usepackage{tikz}
\newcommand\RBox[1]{%
  \tikz\node[draw,rounded corners,align=center,] {#1};%
}
\usepackage{hyperref}
%\usecolortheme{buzz}
%\usecolortheme{wolverine}
%\usetheme{Boadilla}
\usepackage[T1]{fontenc}

\definecolor{mygreen}{rgb}{0,0.6,0}
\definecolor{mygray}{rgb}{0.5,0.5,0.5}
\definecolor{mymauve}{rgb}{0.58,0,0.82}

\lstset{%
  backgroundcolor=\color{white},   % choose the background color; you must add \usepackage{color} or \usepackage{xcolor}
  breakatwhitespace=false,         % sets if automatic breaks should only happen at whitespace
  breaklines=true,                 % sets automatic line breaking
  captionpos=b,                    % sets the caption-position to bottom
  commentstyle=\color{openstack},  % comment style
  extendedchars=true,              % lets you use non-ASCII characters; for 8-bits encodings only, does not work with UTF-8
  keepspaces=true,                 % keeps spaces in text, useful for keeping indentation of code (possibly needs columns=flexible)
  keywordstyle=\color{blue},       % keyword style
%  otherkeywords={*,...},           % if you want to add more keywords to the set
  numbersep=5pt,                   % how far the line-numbers are from the code
  numberstyle=\tiny\color{mygray}, % the style that is used for the line-numbers
  rulecolor=\color{black},         % if not set, the frame-color may be changed on line-breaks within not-black text (e.g. comments (green here))
  showspaces=false,                % show spaces everywhere adding particular underscores; it overrides 'showstringspaces'
  showstringspaces=false,          % underline spaces within strings only
  showtabs=false,                  % show tabs within strings adding particular underscores
  stringstyle=\color{openstack},   % string literal style
}

\setbeamerfont{caption}{series=\normalfont,size=\fontsize{6}{8}}
\setbeamertemplate{caption}{\raggedright\insertcaption\par}

\setlength{\abovecaptionskip}{0pt}
\setlength{\floatsep}{0pt}

\author[Matthew Treinish]{%
    \texorpdfstring{%
        \centering
        Matthew Treinish\\
        Open Source Developer Advocate - IBM \\
        \href{mailto:mtreinish@kortar.org}{mtreinish@kortar.org}\\
        \texttt{mtreinish on Freenode}
   }
   {Matthew Treinish}
}
\date{Oct 7, 2017}

\title[Dirty Clouds Done Dirt Cheap
\hspace{2em}\insertframenumber/\inserttotalframenumber]{Dirty Clouds Done Dirt Cheap}

\begin{document}

{%
\setbeamertemplate{background canvas}{\includegraphics[width=\paperwidth,height=\paperheight]{background_title.png}}
\setbeamertemplate{footline}{}
\begin{frame}[noframenumbering]
    \setbeamercolor{titlelike}{fg=white}
    \setbeamercolor{structure}{fg=white}
    \setbeamercolor{normal text}{fg=white}
    \hypersetup{colorlinks,urlcolor=white}
    \setbeamercolor{author}{fg=white}
    \setbeamercolor{date}{fg=white}
    \setbeamercolor{background}{bg=openstack}
    \titlepage{}
    \centering
    \href{https://github.com/mtreinish/dirty-clouds-done-dirt-cheap/tree/seagl2017}{https://github.com/mtreinish/dirty-clouds-done-dirt-cheap/tree/seagl2017}
\end{frame}
}

\section{Building a Cloud}
\begin{frame}
\frametitle{Building a Cloud}
\centering
\includegraphics[width=.775\textwidth]{cloud_seeding.png}
\end{frame}

\begin{frame}
\frametitle{Scope of the Project}
    \begin{itemize}
        \item Pretend to be a sysadmin with no prior OpenStack knowledge
        \item Try to rely only on install docs and google searches
        \item \$1500 USD Budget
        \item Build a basic compute cloud
        \item Install the Ocata release (from April 2017) from tarballs
        \item No automation or pre-existing install scripts
    \end{itemize}
\end{frame}

\begin{frame}
\frametitle{Buying Hardware}
    \begin{itemize}
        \item Maximize core count per USD
        \item Second priority is amount of RAM per core
        \item Machines don't need to be fast (that costs money!)
    \end{itemize}
\end{frame}

\begin{frame}
    \includegraphics[width=\textwidth]{EBay_logo.png}
\end{frame}

\begin{frame}
    \frametitle{The Servers}
    \begin{tabular}{ l c r }
        \hline
        Model &	PowerEdge R610 \\
        \hline
        Processor &	2x Intel Xeon E5540 \\
        \hline
        Memory Installed & 32GB Total Memory; 8 x 4 GB DDR3 \\
        \hline
        Hard Drives & 2x 146GB 10K SAS Hard Drive \\
        \hline
        RAID Controller & Dell PowerEdge R610 Perc 6i \\
        \hline
        Ethernet & 2x Dual Port Embedded Broadcom NetXtreme ll 5709c \\
        \hline
        Return Policy/Warranty & 60 days Money Back Or Exchange \\
        \hline
    \end{tabular}
    \\
    \begin{center}
        \Huge{\textbf{\$215 Each!!}}
    \end{center}
\end{frame}

\begin{frame}
    \includegraphics[width=\textwidth]{servers_delivered.jpg}  
\end{frame}

\begin{frame}
    \frametitle{LackRack}
    \href{https://wiki.eth0.nl/index.php/LackRack}{https://wiki.eth0.nl/index.php/LackRack}
    \begin{columns}[T]
        \begin{column}{.48\textwidth}
            \begin{itemize}
                \item Use a LACK side table from Ikea
                \item 19 inch width between legs
                \item Can fit 8U
                \item Lots of color choices
                \item \$9.99 USD
            \end{itemize}
        \end{column}
        \begin{column}{.48\textwidth}
            \includegraphics[width=\textwidth]{lackrack_cover.png}
        \end{column}
    \end{columns}
\end{frame}

\begin{frame}
    \centering
    \includegraphics[width=.75\textwidth]{lack_rack.jpg}
\end{frame}

\begin{frame}
    \centering
    \includegraphics[width=.9\textwidth]{data_closet.jpg}
\end{frame}

\begin{frame}
    \frametitle{Quirks with the servers}
    \begin{itemize}
        \item Super stripped down:
            \begin{itemize}
                \item No management interface
                \item No redundant power supply
            \end{itemize}
        \item 4x8GB of RAM not 8x4GB
        \item Memory installed in wrong slots
        \item Dead RAID controller battery
        \item Came with 15k RPM hard drives not 10k RPM
        \item Came pre-installed with Windows Server 2012 (and default password Apple123)
    \end{itemize}
\end{frame}

\section{Installing OpenStack}
\begin{frame}
    \frametitle{Picking OpenStack Services}
    \begin{itemize}
        \item OpenStack has > 50 projects: \href{https://governance.openstack.org/tc/reference/projects/index.html}{https://governance.openstack.org/tc/reference/projects/index.html}
        \item Only want minimal set of projects to spin up VMs
    \end{itemize}
\end{frame}

\begin{frame}
    \frametitle{Compute Starter Kit}
    \begin{columns}[T]
        \begin{column}{.48\textwidth}
            \begin{itemize}
                \item Only need 4 projects to get a functional compute cloud
                \item Documented at:\\ \small \href{https://www.openstack.org/software/sample-configs\#compute-starter-kit}{https://www.openstack.org/software/sample-configs\#compute-starter-kit}
                \item \normalsize The official install guide mostly concerned with these projects
            \end{itemize}
        \end{column}
        \begin{column}{.48\textwidth}
            \includegraphics[height=.25\textheight]{mascots/keystone.eps}
            \includegraphics[height=.25\textheight]{mascots/glance.eps}\\
            \includegraphics[height=.25\textheight]{mascots/nova.eps}
            \includegraphics[height=.25\textheight]{mascots/neutron.eps}\\
        \end{column}
    \end{columns}
\end{frame}

\begin{frame}
    \frametitle{Installing OpenStack}
    \centering
    \includegraphics[width=.8\textwidth]{service-split.eps}
\end{frame}
\begin{frame}
    \frametitle{Installing OpenStack Services}
    \begin{enumerate}
        \item Download service tarball
        \item Create service users
        \item Install binary requirements
        \item Create service dirs in /etc and /var/lib
        \item Copy etc/ from tarball into /etc/\$Service
        \item pip install the tarball
        \item Follow install guide on project configuration and setup
    \end{enumerate}
\end{frame}

\subsection{Installing the Controller}
\subsubsection{Keystone}
\begin{frame}
    \frametitle{Setting Up Keystone}
    \begin{columns}[T]
        \begin{column}{.48\textwidth}
            \begin{itemize}
                \item 2 config options
                \item Install guide doesn't have instructions on
                    apache wsgi app setup
                \item Google search found: \href{https://docs.openstack.org/developer/keystone/apache-httpd.html}{https://docs.openstack.org/developer/keystone/apache-httpd.html}
            \end{itemize}
        \end{column}
        \begin{column}{.48\textwidth}
            \includegraphics[width=\textwidth]{mascots/keystone.eps}
        \end{column}
    \end{columns}
\end{frame}

\begin{frame}
    \frametitle{Python Requirements aren't fun}
    \lstinputlisting[basicstyle=\tiny,language=python]{notes/keystone-start-failure-cut}
\end{frame}

\subsubsection{Glance}
\begin{frame}
    \frametitle{Setting up Glance}
    \begin{columns}[T]
        \begin{column}{.48\textwidth}
            \begin{itemize}
                \item Straightforward configuration:
                    \begin{itemize}
                        \item Auth
                        \item Image Directories
                        \item Database
                    \end{itemize}
            \end{itemize}
        \end{column}
        \begin{column}{.48\textwidth}
            \includegraphics[width=\textwidth]{mascots/glance.eps}
        \end{column}
    \end{columns}
\end{frame}

\begin{frame}
    \frametitle{Don't forget to create glance store directory}
    \lstinputlisting[basicstyle=\tiny,language=python]{notes/glance-store-error-cut}
\end{frame}

\subsubsection{Nova}
\begin{frame}
    \frametitle{Setting up Nova}
    \begin{columns}[T]
        \begin{column}{.48\textwidth}
            \begin{itemize}
                \item Database migrations are slower, took about 3mins
                \item Don't forget the placement API, no docs on apache setup
                \item novnc is problematic from source
                \item Set \textit{force\_config\_drive} option to true
            \end{itemize}
        \end{column}
        \begin{column}{.48\textwidth}
            \includegraphics[width=\textwidth]{mascots/nova.eps}
        \end{column}
    \end{columns}
\end{frame}

\begin{frame}
    \frametitle{Requirements still aren't fun}
    \lstinputlisting[basicstyle=\tiny,language=python]{notes/nova-requirements-error-cut}
\end{frame}

\begin{frame}
    \frametitle{You need a  sudoers file}
    \lstinputlisting[basicstyle=\tiny,language=python]{notes/nova-sudo-error-cut}
\end{frame}

\begin{frame}
    \frametitle{Don't forget to create state directories}
    \lstinputlisting[basicstyle=\tiny,language=python]{notes/nova-statedir-error-cut}
\end{frame}

\subsubsection{Neutron}
\begin{frame}
    \frametitle{Networking Configuration}
    \centering
    \includegraphics[width=.7\textwidth]{deploy-lb-provider-overview.png}
\end{frame}

\begin{frame}
    \frametitle{Setting Up Neutron}
    \begin{columns}[T]
        \begin{column}{.48\textwidth}
            \begin{itemize}
                \item Too many configuration files
                \item Blindly copying pasting from install guide
                \item Rootwrap and sudo configuration are not documented
                \item First time I had to look at packages and/or devstack
            \end{itemize}
        \end{column}
        \begin{column}{.48\textwidth}
            \includegraphics[width=\textwidth]{mascots/neutron.eps}
        \end{column}
    \end{columns}
\end{frame}

\begin{frame}
    \frametitle{Any guesses what this means}
    \lstinputlisting[basicstyle=\tiny,language=python]{notes/neutron-linuxbridge-error-cut}
\end{frame}

\begin{frame}
    \frametitle{DHCP Fun}
    \centering
    \includegraphics[width=.7\textwidth]{network-topology-with-router.png}
\end{frame}

\subsection{Setting up Compute Nodes}
\begin{frame}
    \frametitle{Setting up Compute Nodes}
    \begin{itemize}
        \item Same basic formula
        \item Rinse and repeat 4 times
        \item Don't forget to run nova discover\_hosts for each new node
    \end{itemize}
\end{frame}


\subsection{Booting the first server}
%Blank for dramatic effect
\begin{frame}
    
\end{frame}

\begin{frame}
    \frametitle{Blank Images}
    \lstinputlisting[basicstyle=\small,language=python]{notes/glance-store-0-bytes}
\end{frame}

\begin{frame}
    \frametitle{Where's my metadata?}
    \begin{itemize}
        \item Without DHCP neutron can't set routes to metadata service at \textit{169.254.169.254}
        \item Need to get IP address into the guest to start networking
        \item Use Nova's ConfigDrive works except most versions of Cloud Init don't set static IP from config drives
        \item Need to create custom images with OpenStack Glean
    \end{itemize}
\end{frame}



\section{What to do with your budget cloud?}
\begin{frame}
    \frametitle{What to do with your budget cloud?}
    \centering
    \includegraphics[width=.85\textwidth]{futurama-fry.png}
\end{frame}

\subsection{OpenStack Development}
\begin{frame}
    \frametitle{OpenStack Development}
    \begin{itemize}
        \item Lots of capacity for running devstack
        \item A really good platform to develop and test OpenStack
              applications
        \item For example I found 4 tempest bugs testing it on the cloud
    \end{itemize}
\end{frame}

\subsection{Cloud Native Compute Workloads}
\begin{frame}
        \frametitle{Cloud Native Compute Workloads}
        \begin{itemize}
        \item Good for running embarrasingly parallel workloads
        \item Each invidiual machine is slow, but a fair amount of
            parallel capacity.
        \item My example use case: \href{https://github.com/mtreinish/handbrakecloud}{https://github.com/mtreinish/handbrakecloud}
    \end{itemize}
\end{frame}

\subsection{Virtualized Infrastructure}
\begin{frame}
    \frametitle{Virtualized Home Infrastructure}
    \begin{itemize}
        \item Gives you the flexability
        \item Also enables HA migration to public cloud
    \end{itemize}
\end{frame}


\section{Conclusion}
\subsection{Installation Pain Points}
\begin{frame}
    \frametitle{Installation Pain Points}
    \begin{itemize}
        \item Python Packaging:
            \begin{itemize}
                \item Binary Dependencies
                \item etc files (and any data files)
                \item No dependency solver, always use upper constraints
            \end{itemize}
        \item Debugging OpenStack requires a high level of competence
    \end{itemize}
\end{frame}
\begin{frame}
    \frametitle{Making OpenStack Better for Small Deployments}
    \begin{itemize}
        \item Honestly, it's not that bad
        \item >=90\% of the issues were because of using tarballs
        \item Networking and neutron is too confusing
        \item Work on improving logging and error reporting
    \end{itemize}
\end{frame}

\subsection{Why you don't want to do this}
\begin{frame}
    \frametitle{Why you don't want to do this}
    \begin{itemize}
        \item 5x 1U Servers in your bedroom closet is not pleasant
        \item The power bill (at peak draw it's about 1.1kW for the rack)
        \item Don't get to spend \$1,328.37 on a weekend vacation
    \end{itemize}
\end{frame}


\subsection{More Information}
\begin{frame}
\frametitle{Where to get more information}
    \begin{itemize}
        \item openstack-dev ML\: \href{mailto:openstack-dev@lists.openstack.org}{openstack-dev@lists.openstack.org}
        \item Ocata install guides\: \href{https://docs.openstack.org/project-install-guide/ocata/}{https://docs.openstack.org/project-install-guide/ocata/}
        \item Ocata network guides\: \href{https://docs.openstack.org/ocata/networking-guide/}{https://docs.openstack.org/ocata/networking-guide/}
        \item Blog post about the project: \href{http://blog.kortar.org/?p=380}{http://blog.kortar.org/?p=380}
   \end{itemize}
\end{frame}


\end{document}
